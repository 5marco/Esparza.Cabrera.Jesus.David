\documentclass[12pt]{article}


\usepackage[utf8]{inputenc}

\title{\textbf{CARACTERÍSTICAS DE LOS CONVERTIDORES DE POTENCIA CA-CD, CD-CA, CA-CA y CD-CD}}

\author{Esparza Cabrera Jesús David\\
		17 de Septiembre del 2019\\}
\date{}
\begin{document}

\maketitle

\section{Convertidores CA-CD}

Un convertidor de corriente alterna a corriente directa parte de un rectificador de
onda completa. Su carga puede ser puramente resistiva. La forma de onda de
salida del rectificador. Al agregarle a este rectificador un
capacitor en paralelo, el convertidor se comporta como un
filtro ya que se produce un voltaje a la salida que es esencialmente continuo. 

El convertidor CA-CD nos proporciona una señal de salida rectificada (casi
constante) de valor V m , donde V m es igual al valor pico del voltaje de entrada.
Este
valor se puede considerar muy pequeño y de esta manera encontrar el valor del resistor y
del capacitor para un valor de voltaje directo deseado.

\section{Convertidores CD-CA}
Los convertidores de corriente directa a corriente alterna son utilizados como
drivers de motores y como fuentes de corriente alterna ininterrumpida y tienen como
objetivo producir una señal de corriente alterna sinusoidal, cuya magnitud y frecuencia
puedan ser controladas. Existen diversos tipos de convertidores inversores de los
cuales el convertidor de una sola pierna y el convertidor en puente de media y onda
completa.
Estas topologías generan una señal de salida que oscila entre el valor constante +V d /2 y –
V d /2 para medio puente y una sola pierna, y +V d y –V d para puente completo, figura 1.3. A
pesar de que esta señal es cuadrada, posteriormente se filtra para obtener una señal
senoidal.
En las topologías antes mencionadas existe lo que se conoce como conmutación
imperfecta, la cual es uno de los mayores contribuyentes a la pérdida de potencia en estos
convertidores. Estos dispositivos de conmutación absorben potencia cuando sus
interruptores se encienden o apagan, si la transición se produce cuando tanto el voltaje
como la corriente son diferentes de cero. Si se aumenta la frecuencia de conmutación, estas
transiciones suceden con mayor frecuencia y la pérdida de potencia media en los interruptores aumenta. En muchas ocasiones el uso de altas frecuencias es deseable por la
reducción significativa en tamaño de muchos dispositivos [7].
Para poder realizar conmutaciones de más altas frecuencias en los convertidores, las
pérdidas anteriormente mencionadas se pueden minimizar si cada interruptor en el
convertidor cambia su estado (de encendido a apagado o viceversa) cuando el voltaje a
través de él y/o la corriente es cero durante el instante de conmutación.
\section{Convertidores CA-CA}
Los convertidores de CA a CA son más complicados que los convertidores de CA a CC porque la conversión de CA requiere un cambio de tensión, frecuencia y capacidades de bloqueo de tensión bipolar, lo cual generalmente requiere tipologías de dispositivos complejas. Los convertidores que tienen fundamentalmente las mismas frecuencias de entrada y salida se llaman “Controladores de CA” y la conversión es de tensión fija frecuencia fija a tensión variable frecuencia fija (como aplicaciones de atenuadores de luz y control de motores de CA monofásicos que son típicamente usados en los electrodomésticos). Los convertidores que convierten una tensión fija frecuencia fija a tensión variable frecuencia variable se les llama “cicloconvertidores”. Estos se utilizan en aplicaciones de alta potencia que accionan motores de inducción y síncronos, generalmente controlados por fase y tradicionalmente utilizan tiristores debido a su facilidad de conmutación de fase. Otra forma de lograr la conversión CA/CA es utilizando CA/CC y CC/CA a través de un enlace CC intermedio.
\section{Convertidores CD-CD}
En muchas aplicaciones industriales, es necesario el convertir una fuente de
poder de corriente directa (CD) de voltaje fijo a una fuente de CD de voltaje variable. Un
convertidor de CD, convierte directamente de CD a CD. Este convertidor se puede considerar
como el equivalente a un transformador de corriente alterna (CA) con una relación de vueltas
que varía en forma continua. Al igual que un transformador, puede utilizarse como una fuente
de CD reductora o elevadora de voltaje [1].
Los convertidores CD-CD se utilizan ampliamente en el control de los motores de tracción de
automóviles eléctricos, tranvías eléctricos, grúas marinas, montacargas y elevadores de minas.
En lo que a nosotros nos concierne el convertidor CD-CD se utilizará en la primera etapa del
balastro para la corrección del factor de potencia y obtener una salida en CD estable para
alimentar el inversor resonante el cual trabajará en alta frecuencia. En este capítulo se
analizarán 3 topologías de convertidores CD-CD las cuales son: Topología Elevadora,
Reductora-elevadora y Flyback.




\end{document}
